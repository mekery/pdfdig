% Generated by Sphinx.
\def\sphinxdocclass{report}
\documentclass[letterpaper,10pt,oneside]{sphinxmanual}
\usepackage[utf8]{inputenc}
\DeclareUnicodeCharacter{00A0}{\nobreakspace}
\usepackage[T1]{fontenc}
\usepackage[english]{babel}
\usepackage{times}
\usepackage[Bjarne]{fncychap}
\usepackage{longtable}
\usepackage{sphinx}
\usepackage{multirow}


\title{PDFDig Documentation}
\date{July 05, 2012}
\release{0.2}
\author{Micle Bu}
\newcommand{\sphinxlogo}{}
\renewcommand{\releasename}{Release}
\makeindex

\makeatletter
\def\PYG@reset{\let\PYG@it=\relax \let\PYG@bf=\relax%
    \let\PYG@ul=\relax \let\PYG@tc=\relax%
    \let\PYG@bc=\relax \let\PYG@ff=\relax}
\def\PYG@tok#1{\csname PYG@tok@#1\endcsname}
\def\PYG@toks#1+{\ifx\relax#1\empty\else%
    \PYG@tok{#1}\expandafter\PYG@toks\fi}
\def\PYG@do#1{\PYG@bc{\PYG@tc{\PYG@ul{%
    \PYG@it{\PYG@bf{\PYG@ff{#1}}}}}}}
\def\PYG#1#2{\PYG@reset\PYG@toks#1+\relax+\PYG@do{#2}}

\expandafter\def\csname PYG@tok@gd\endcsname{\def\PYG@tc##1{\textcolor[rgb]{0.63,0.00,0.00}{##1}}}
\expandafter\def\csname PYG@tok@gu\endcsname{\let\PYG@bf=\textbf\def\PYG@tc##1{\textcolor[rgb]{0.50,0.00,0.50}{##1}}}
\expandafter\def\csname PYG@tok@gt\endcsname{\def\PYG@tc##1{\textcolor[rgb]{0.00,0.25,0.82}{##1}}}
\expandafter\def\csname PYG@tok@gs\endcsname{\let\PYG@bf=\textbf}
\expandafter\def\csname PYG@tok@gr\endcsname{\def\PYG@tc##1{\textcolor[rgb]{1.00,0.00,0.00}{##1}}}
\expandafter\def\csname PYG@tok@cm\endcsname{\let\PYG@it=\textit\def\PYG@tc##1{\textcolor[rgb]{0.25,0.50,0.56}{##1}}}
\expandafter\def\csname PYG@tok@vg\endcsname{\def\PYG@tc##1{\textcolor[rgb]{0.73,0.38,0.84}{##1}}}
\expandafter\def\csname PYG@tok@m\endcsname{\def\PYG@tc##1{\textcolor[rgb]{0.13,0.50,0.31}{##1}}}
\expandafter\def\csname PYG@tok@mh\endcsname{\def\PYG@tc##1{\textcolor[rgb]{0.13,0.50,0.31}{##1}}}
\expandafter\def\csname PYG@tok@cs\endcsname{\def\PYG@tc##1{\textcolor[rgb]{0.25,0.50,0.56}{##1}}\def\PYG@bc##1{\setlength{\fboxsep}{0pt}\colorbox[rgb]{1.00,0.94,0.94}{\strut ##1}}}
\expandafter\def\csname PYG@tok@ge\endcsname{\let\PYG@it=\textit}
\expandafter\def\csname PYG@tok@vc\endcsname{\def\PYG@tc##1{\textcolor[rgb]{0.73,0.38,0.84}{##1}}}
\expandafter\def\csname PYG@tok@il\endcsname{\def\PYG@tc##1{\textcolor[rgb]{0.13,0.50,0.31}{##1}}}
\expandafter\def\csname PYG@tok@go\endcsname{\def\PYG@tc##1{\textcolor[rgb]{0.19,0.19,0.19}{##1}}}
\expandafter\def\csname PYG@tok@cp\endcsname{\def\PYG@tc##1{\textcolor[rgb]{0.00,0.44,0.13}{##1}}}
\expandafter\def\csname PYG@tok@gi\endcsname{\def\PYG@tc##1{\textcolor[rgb]{0.00,0.63,0.00}{##1}}}
\expandafter\def\csname PYG@tok@gh\endcsname{\let\PYG@bf=\textbf\def\PYG@tc##1{\textcolor[rgb]{0.00,0.00,0.50}{##1}}}
\expandafter\def\csname PYG@tok@ni\endcsname{\let\PYG@bf=\textbf\def\PYG@tc##1{\textcolor[rgb]{0.84,0.33,0.22}{##1}}}
\expandafter\def\csname PYG@tok@nl\endcsname{\let\PYG@bf=\textbf\def\PYG@tc##1{\textcolor[rgb]{0.00,0.13,0.44}{##1}}}
\expandafter\def\csname PYG@tok@nn\endcsname{\let\PYG@bf=\textbf\def\PYG@tc##1{\textcolor[rgb]{0.05,0.52,0.71}{##1}}}
\expandafter\def\csname PYG@tok@no\endcsname{\def\PYG@tc##1{\textcolor[rgb]{0.38,0.68,0.84}{##1}}}
\expandafter\def\csname PYG@tok@na\endcsname{\def\PYG@tc##1{\textcolor[rgb]{0.25,0.44,0.63}{##1}}}
\expandafter\def\csname PYG@tok@nb\endcsname{\def\PYG@tc##1{\textcolor[rgb]{0.00,0.44,0.13}{##1}}}
\expandafter\def\csname PYG@tok@nc\endcsname{\let\PYG@bf=\textbf\def\PYG@tc##1{\textcolor[rgb]{0.05,0.52,0.71}{##1}}}
\expandafter\def\csname PYG@tok@nd\endcsname{\let\PYG@bf=\textbf\def\PYG@tc##1{\textcolor[rgb]{0.33,0.33,0.33}{##1}}}
\expandafter\def\csname PYG@tok@ne\endcsname{\def\PYG@tc##1{\textcolor[rgb]{0.00,0.44,0.13}{##1}}}
\expandafter\def\csname PYG@tok@nf\endcsname{\def\PYG@tc##1{\textcolor[rgb]{0.02,0.16,0.49}{##1}}}
\expandafter\def\csname PYG@tok@si\endcsname{\let\PYG@it=\textit\def\PYG@tc##1{\textcolor[rgb]{0.44,0.63,0.82}{##1}}}
\expandafter\def\csname PYG@tok@s2\endcsname{\def\PYG@tc##1{\textcolor[rgb]{0.25,0.44,0.63}{##1}}}
\expandafter\def\csname PYG@tok@vi\endcsname{\def\PYG@tc##1{\textcolor[rgb]{0.73,0.38,0.84}{##1}}}
\expandafter\def\csname PYG@tok@nt\endcsname{\let\PYG@bf=\textbf\def\PYG@tc##1{\textcolor[rgb]{0.02,0.16,0.45}{##1}}}
\expandafter\def\csname PYG@tok@nv\endcsname{\def\PYG@tc##1{\textcolor[rgb]{0.73,0.38,0.84}{##1}}}
\expandafter\def\csname PYG@tok@s1\endcsname{\def\PYG@tc##1{\textcolor[rgb]{0.25,0.44,0.63}{##1}}}
\expandafter\def\csname PYG@tok@gp\endcsname{\let\PYG@bf=\textbf\def\PYG@tc##1{\textcolor[rgb]{0.78,0.36,0.04}{##1}}}
\expandafter\def\csname PYG@tok@sh\endcsname{\def\PYG@tc##1{\textcolor[rgb]{0.25,0.44,0.63}{##1}}}
\expandafter\def\csname PYG@tok@ow\endcsname{\let\PYG@bf=\textbf\def\PYG@tc##1{\textcolor[rgb]{0.00,0.44,0.13}{##1}}}
\expandafter\def\csname PYG@tok@sx\endcsname{\def\PYG@tc##1{\textcolor[rgb]{0.78,0.36,0.04}{##1}}}
\expandafter\def\csname PYG@tok@bp\endcsname{\def\PYG@tc##1{\textcolor[rgb]{0.00,0.44,0.13}{##1}}}
\expandafter\def\csname PYG@tok@c1\endcsname{\let\PYG@it=\textit\def\PYG@tc##1{\textcolor[rgb]{0.25,0.50,0.56}{##1}}}
\expandafter\def\csname PYG@tok@kc\endcsname{\let\PYG@bf=\textbf\def\PYG@tc##1{\textcolor[rgb]{0.00,0.44,0.13}{##1}}}
\expandafter\def\csname PYG@tok@c\endcsname{\let\PYG@it=\textit\def\PYG@tc##1{\textcolor[rgb]{0.25,0.50,0.56}{##1}}}
\expandafter\def\csname PYG@tok@mf\endcsname{\def\PYG@tc##1{\textcolor[rgb]{0.13,0.50,0.31}{##1}}}
\expandafter\def\csname PYG@tok@err\endcsname{\def\PYG@bc##1{\setlength{\fboxsep}{0pt}\fcolorbox[rgb]{1.00,0.00,0.00}{1,1,1}{\strut ##1}}}
\expandafter\def\csname PYG@tok@kd\endcsname{\let\PYG@bf=\textbf\def\PYG@tc##1{\textcolor[rgb]{0.00,0.44,0.13}{##1}}}
\expandafter\def\csname PYG@tok@ss\endcsname{\def\PYG@tc##1{\textcolor[rgb]{0.32,0.47,0.09}{##1}}}
\expandafter\def\csname PYG@tok@sr\endcsname{\def\PYG@tc##1{\textcolor[rgb]{0.14,0.33,0.53}{##1}}}
\expandafter\def\csname PYG@tok@mo\endcsname{\def\PYG@tc##1{\textcolor[rgb]{0.13,0.50,0.31}{##1}}}
\expandafter\def\csname PYG@tok@mi\endcsname{\def\PYG@tc##1{\textcolor[rgb]{0.13,0.50,0.31}{##1}}}
\expandafter\def\csname PYG@tok@kn\endcsname{\let\PYG@bf=\textbf\def\PYG@tc##1{\textcolor[rgb]{0.00,0.44,0.13}{##1}}}
\expandafter\def\csname PYG@tok@o\endcsname{\def\PYG@tc##1{\textcolor[rgb]{0.40,0.40,0.40}{##1}}}
\expandafter\def\csname PYG@tok@kr\endcsname{\let\PYG@bf=\textbf\def\PYG@tc##1{\textcolor[rgb]{0.00,0.44,0.13}{##1}}}
\expandafter\def\csname PYG@tok@s\endcsname{\def\PYG@tc##1{\textcolor[rgb]{0.25,0.44,0.63}{##1}}}
\expandafter\def\csname PYG@tok@kp\endcsname{\def\PYG@tc##1{\textcolor[rgb]{0.00,0.44,0.13}{##1}}}
\expandafter\def\csname PYG@tok@w\endcsname{\def\PYG@tc##1{\textcolor[rgb]{0.73,0.73,0.73}{##1}}}
\expandafter\def\csname PYG@tok@kt\endcsname{\def\PYG@tc##1{\textcolor[rgb]{0.56,0.13,0.00}{##1}}}
\expandafter\def\csname PYG@tok@sc\endcsname{\def\PYG@tc##1{\textcolor[rgb]{0.25,0.44,0.63}{##1}}}
\expandafter\def\csname PYG@tok@sb\endcsname{\def\PYG@tc##1{\textcolor[rgb]{0.25,0.44,0.63}{##1}}}
\expandafter\def\csname PYG@tok@k\endcsname{\let\PYG@bf=\textbf\def\PYG@tc##1{\textcolor[rgb]{0.00,0.44,0.13}{##1}}}
\expandafter\def\csname PYG@tok@se\endcsname{\let\PYG@bf=\textbf\def\PYG@tc##1{\textcolor[rgb]{0.25,0.44,0.63}{##1}}}
\expandafter\def\csname PYG@tok@sd\endcsname{\let\PYG@it=\textit\def\PYG@tc##1{\textcolor[rgb]{0.25,0.44,0.63}{##1}}}

\def\PYGZbs{\char`\\}
\def\PYGZus{\char`\_}
\def\PYGZob{\char`\{}
\def\PYGZcb{\char`\}}
\def\PYGZca{\char`\^}
\def\PYGZam{\char`\&}
\def\PYGZlt{\char`\<}
\def\PYGZgt{\char`\>}
\def\PYGZsh{\char`\#}
\def\PYGZpc{\char`\%}
\def\PYGZdl{\char`\$}
\def\PYGZti{\char`\~}
% for compatibility with earlier versions
\def\PYGZat{@}
\def\PYGZlb{[}
\def\PYGZrb{]}
\makeatother

\begin{document}

\maketitle
\tableofcontents
\phantomsection\label{index::doc}\phantomsection\label{index:contents}



\chapter{Introduction}
\label{intro:introduction}\label{intro:pdfdig-documentation}\label{intro::doc}
PDFDig is a useful tool to dig content from pdf document, which is based on pdftotext \footnote{
pdftotext: \href{http://en.wikipedia.org/wiki/Pdftotext}{http://en.wikipedia.org/wiki/Pdftotext}
} and PDFMiner \footnote{
PDFMiner: \href{http://www.unixuser.org/~euske/python/pdfminer/}{http://www.unixuser.org/\textasciitilde{}euske/python/pdfminer/}
}.


\section{Features}
\label{intro:features}\begin{itemize}
\item {} 
Convert pdf to txt.

\item {} 
Search in pdf document, working like grep.

\item {} 
Build table of content(TOC) of pdf document.

\item {} 
Get pdf metadata.

\end{itemize}
\paragraph{References}


\chapter{PDFDig Tutorial}
\label{tutorial::doc}\label{tutorial:pdfdig-tutorial}
This tutorial serves as the quick start for PDFDig.


\section{Prerequisites}
\label{tutorial:prerequisites}

\subsection{Python}
\label{tutorial:python}
PDFDig is written in Python, so you should prepare Python environment first. Both Python 2 and Python 3 are OK.
\begin{description}
\item[{Download Python from:}] \leavevmode\begin{itemize}
\item {} 
\href{http://www.python.org/getit/}{http://www.python.org/getit/}

\end{itemize}

\end{description}

Since PDFDig only provides Command Line Interface(CLI) utilities currently, we strongly recommand Windows users to use \href{http://www.cygwin.com/}{Cygwin}, a linux-like environment for Windows, as running environment for PDFDig to get full features of PDFDig.


\subsection{Cygwin}
\label{tutorial:id1}
\href{http://cygwin.com/install.html}{Installing Cygwin} is pretty easy and straightforward.
\begin{itemize}
\item {} 
Download \href{http://cygwin.com/setup.exe}{setup.exe}.

\item {} 
Run setup.exe and follow its navigation.

\item {} 
When setup.exe asks you to \textbf{Select Packages}, make sure you have selected \textbf{Python Default} and then \textbf{python: Python language interpreter}.

\item {} 
After installation, you may try \emph{python --version} within Cygwin Terminal.

\end{itemize}


\subsection{pdftotext}
\label{tutorial:pdftotext}
PDFDig does not extract content from PDF documents directly by itself, but use a efficient utility, pdftotext, which is freely available and included by default with many Linux distributions. Xpdf provides a pdftotext port to Windows platform.
\begin{description}
\item[{\textbf{Windows users:}}] \leavevmode\begin{itemize}
\item {} 
Download xpdf binaries, looks like \textbf{xpdfbin-win-3.03.zip}, from: \href{http://www.foolabs.com/xpdf/download.html}{http://www.foolabs.com/xpdf/download.html}

\item {} 
Extract xpdfbin-win-3.03.zip in your favorite directory, take D:\textbackslash{} as an example, you'll get D:\textbackslash{}xpdfbin-win-3.03.

\item {} 
pdftotext.exe locates in D:\textbackslash{}xpdfbin-win-3.03\textbackslash{}bin32\textbackslash{} or D:\textbackslash{}xpdfbin-win-3.03\textbackslash{}bin64\textbackslash{}

\item {} 
Choose correct version of pdftotext depending on your system architecture, take 32-bit system as an example, you should use the pdftotext.exe in D:\textbackslash{}xpdfbin-win-3.03\textbackslash{}bin32\textbackslash{}.

\item {} 
Add pdftotext.exe directory \textbf{D:\textbackslash{}xpdfbin-win-3.03\textbackslash{}bin32\textbackslash{}} to PATH environment variable to ensure system can find pdftotext.exe.

\end{itemize}

\end{description}

\textbf{Unix/Linux users:}

pdftotext is available and included by default with many Linux distributions. If pdftotext does not exsit in your system, install poppler-utils package.

\textbf{Check Test:}

You can test pdftotext, just run

\begin{Verbatim}[commandchars=\\\{\}]
\$ pdftotext -v
Copyright 2005-2011 The Poppler Developers - http://poppler.freedesktop.org
Copyright 1996-2004 Glyph \& Cog, LLC
\end{Verbatim}


\section{Installation}
\label{tutorial:installation}\begin{enumerate}
\item {} 
Get PDFDig source, the tarball file looks like pdfdig-1.0.tar.bz2.

\item {} 
Extract the tarball file.

\end{enumerate}

\begin{Verbatim}[commandchars=\\\{\}]
\$ tar jxvf pdfdig-1.0.tar.bz2
\end{Verbatim}
\begin{enumerate}
\setcounter{enumi}{2}
\item {} 
Install PDFDig.

\end{enumerate}

\begin{Verbatim}[commandchars=\\\{\}]
\$ cd pdfdig-1.0
\$ sudo python setup.py install
\end{Verbatim}

After the installation, PDFDig will copy PDFDig library to Python library and install 4 executable utilities in your system.


\section{Utility}
\label{tutorial:utility}
PDFDig provides you 4 Command Line Interface(CLI) utilities, helps to process PDF documents and do the text processing in command line, so you may need a terminal in Unix/Linux or run \emph{cmd} in Windows before using these utilities.

Refer to {\hyperref[utility::doc]{\emph{PDFDig Utility}}} for details.


\chapter{PDFDig Utility}
\label{utility::doc}\label{utility:pdfdig-utility}
PDFDig provides you 4 Command Line Interface(CLI) utilities, helps to process PDF documents and do the text processing in command line, so you may need a terminal in Unix/Linux or run \emph{cmd} in Windows before using these utilities.


\section{1. pdftotext.py}
\label{utility:pdftotext-py}
pdftotext.py converts pdf to text. There are tens of similar utilities can do this job, while few of them, including pdftotext, can process line-break, hyphen and extra white spaces appropriately, and some of them render the pdf in physical order, which are unsuitable for multi-column pdf documents.

pdftotext.py uses pdftotext to get the text content of pdf document, then normalizes the text content.


\subsection{Usage}
\label{utility:usage}
\begin{Verbatim}[commandchars=\\\{\}]
\$ pdftotext.py [options] filename1 ...
\end{Verbatim}

The \textbf{pdftotext.py} script has several options:
\index{pdftotext.py command line option!-o, --output OUTPUTFILE}\index{-o, --output OUTPUTFILE!pdftotext.py command line option}

\begin{fulllineitems}
\phantomsection\label{utility:cmdoption-pdftotext.py-o}\pysigline{\bfcode{-o}\code{}\code{,~}\bfcode{--output}\code{~OUTPUTFILE}}
Specify the output file. e.g: output.txt

\end{fulllineitems}

\index{pdftotext.py command line option!-y, --layout LAYOUT}\index{-y, --layout LAYOUT!pdftotext.py command line option}

\begin{fulllineitems}
\phantomsection\label{utility:cmdoption-pdftotext.py-y}\pysigline{\bfcode{-y}\code{}\code{,~}\bfcode{--layout}\code{~LAYOUT}}
Maintain the layout of the text. LAYOUT can be:
\begin{description}
\item[{\textbf{raw}}] \leavevmode
keep the text in content stream order. This is the default setting.

\item[{\textbf{layout}}] \leavevmode
preserve the original physical layout of the text.

\end{description}

\end{fulllineitems}

\index{pdftotext.py command line option!-f, --first-page INT}\index{-f, --first-page INT!pdftotext.py command line option}

\begin{fulllineitems}
\phantomsection\label{utility:cmdoption-pdftotext.py-f}\pysigline{\bfcode{-f}\code{}\code{,~}\bfcode{--first-page}\code{~INT}}
First page to convert.

\end{fulllineitems}

\index{pdftotext.py command line option!-l, --last-page INT}\index{-l, --last-page INT!pdftotext.py command line option}

\begin{fulllineitems}
\phantomsection\label{utility:cmdoption-pdftotext.py-l}\pysigline{\bfcode{-l}\code{}\code{,~}\bfcode{--last-page}\code{~INT}}
Last page to convert.

\end{fulllineitems}

\index{pdftotext.py command line option!-p, --page INT}\index{-p, --page INT!pdftotext.py command line option}

\begin{fulllineitems}
\phantomsection\label{utility:cmdoption-pdftotext.py-p}\pysigline{\bfcode{-p}\code{}\code{,~}\bfcode{--page}\code{~INT}}
Specify a page to convert.

\end{fulllineitems}

\index{pdftotext.py command line option!-h, --help}\index{-h, --help!pdftotext.py command line option}

\begin{fulllineitems}
\phantomsection\label{utility:cmdoption-pdftotext.py-h}\pysigline{\bfcode{-h}\code{}\code{,~}\bfcode{--help}\code{}}
Print usage information.

\end{fulllineitems}



\subsection{Examples}
\label{utility:examples}
\begin{Verbatim}[commandchars=\\\{\}]
\$ pdftotext.py input.pdf
\$ pdftotext.py -o output.txt input.pdf
\end{Verbatim}


\section{2. pdfgrep.py}
\label{utility:pdfgrep-py}
pdfgrep.py enables you to search and count in pdf files. pdfgrep.py searches in grep-style, which means you can use regular expression in search and get matching lines.


\subsection{Usage}
\label{utility:id1}
\begin{Verbatim}[commandchars=\\\{\}]
\$ pdfgrep.py [options] pattern filename ...
\end{Verbatim}

The \textbf{pdfgrep.py} script has several options:
\index{pdfgrep.py command line option!-o, --output OUTPUTFILE}\index{-o, --output OUTPUTFILE!pdfgrep.py command line option}

\begin{fulllineitems}
\phantomsection\label{utility:cmdoption-pdfgrep.py-o}\pysigline{\bfcode{-o}\code{}\code{,~}\bfcode{--output}\code{~OUTPUTFILE}}
Specify the output file. e.g: output.txt

\end{fulllineitems}

\index{pdfgrep.py command line option!-c, --count}\index{-c, --count!pdfgrep.py command line option}

\begin{fulllineitems}
\phantomsection\label{utility:cmdoption-pdfgrep.py-c}\pysigline{\bfcode{-c}\code{}\code{,~}\bfcode{--count}\code{}}
Print the number of matches for each input file, instead of normal ouput.

\end{fulllineitems}

\index{pdfgrep.py command line option!-i, --ignore-case}\index{-i, --ignore-case!pdfgrep.py command line option}

\begin{fulllineitems}
\phantomsection\label{utility:cmdoption-pdfgrep.py-i}\pysigline{\bfcode{-i}\code{}\code{,~}\bfcode{--ignore-case}\code{}}
Ingnore case distinctions.

\end{fulllineitems}

\index{pdfgrep.py command line option!-f, --file-prefix}\index{-f, --file-prefix!pdfgrep.py command line option}

\begin{fulllineitems}
\phantomsection\label{utility:cmdoption-pdfgrep.py-f}\pysigline{\bfcode{-f}\code{}\code{,~}\bfcode{--file-prefix}\code{}}
Prefix each line of output with input file.

\end{fulllineitems}

\index{pdfgrep.py command line option!-p, --page-number}\index{-p, --page-number!pdfgrep.py command line option}

\begin{fulllineitems}
\phantomsection\label{utility:cmdoption-pdfgrep.py-p}\pysigline{\bfcode{-p}\code{}\code{,~}\bfcode{--page-number}\code{}}
Prefix each line of output with page number.

\end{fulllineitems}

\index{pdfgrep.py command line option!-n, --line-number}\index{-n, --line-number!pdfgrep.py command line option}

\begin{fulllineitems}
\phantomsection\label{utility:cmdoption-pdfgrep.py-n}\pysigline{\bfcode{-n}\code{}\code{,~}\bfcode{--line-number}\code{}}
Prefix each line of output with 1-based line number within its txt file.

\end{fulllineitems}

\index{pdfgrep.py command line option!-t, --context NUM}\index{-t, --context NUM!pdfgrep.py command line option}

\begin{fulllineitems}
\phantomsection\label{utility:cmdoption-pdfgrep.py-t}\pysigline{\bfcode{-t}\code{}\code{,~}\bfcode{--context}\code{~NUM}}
Print at most NUM characters of context around each match. e.g: -t 100

\end{fulllineitems}

\index{pdfgrep.py command line option!-d, --dictionary PATH}\index{-d, --dictionary PATH!pdfgrep.py command line option}

\begin{fulllineitems}
\phantomsection\label{utility:cmdoption-pdfgrep.py-d}\pysigline{\bfcode{-d}\code{}\code{,~}\bfcode{--dictionary}\code{~PATH}}
Specify the TOC dictionary directory.

\end{fulllineitems}

\index{pdfgrep.py command line option!-l, --location}\index{-l, --location!pdfgrep.py command line option}

\begin{fulllineitems}
\phantomsection\label{utility:cmdoption-pdfgrep.py-l}\pysigline{\bfcode{-l}\code{}\code{,~}\bfcode{--location}\code{}}
Print the match location within TOC.

\end{fulllineitems}

\index{pdfgrep.py command line option!-C, --color COLOR}\index{-C, --color COLOR!pdfgrep.py command line option}

\begin{fulllineitems}
\phantomsection\label{utility:cmdoption-pdfgrep.py-C}\pysigline{\bfcode{-C}\code{}\code{,~}\bfcode{--color}\code{~COLOR}}
Highlight color. COLOR is red by default, also can be black,red,green,orange,blue,purple,bluegreen or white.

\end{fulllineitems}

\index{pdfgrep.py command line option!-h, --help}\index{-h, --help!pdfgrep.py command line option}

\begin{fulllineitems}
\phantomsection\label{utility:cmdoption-pdfgrep.py-h}\pysigline{\bfcode{-h}\code{}\code{,~}\bfcode{--help}\code{}}
Print usage information.

\end{fulllineitems}



\subsection{Examples}
\label{utility:id2}
\begin{Verbatim}[commandchars=\\\{\}]
\# search in pdf, support multi-pdf at once
\$ pdfgrep.py -in "keword" input1.pdf input2.pdf

\# search in directory
\$ pdfgrep.py -in "keword" pdf-directory

\# search and count
\$ pdfgrep.py -c "keword" input.pdf

\# support location within TOC
\$ pdfgrep.py -nl -o output.txt input.pdf

\# change highlight color
\$ pdfgrep.py -C blue output.txt input.pdf

\# save results in a file with a name of output.txt, highlight doesn't work in this case
\$ pdfgrep.py -nl -o output.txt input.pdf
\end{Verbatim}


\subsection{Output Formarts}
\label{utility:output-formarts}
The output of search results are formatted to make it more readable. For example, run

\begin{Verbatim}[commandchars=\\\{\}]
\$ pdfgrep.py -inf  "brain" input.pdf
\end{Verbatim}

The output may look like:

\begin{Verbatim}[commandchars=\\\{\}]
@F:input.pdf @N: 335   @C:  Longitudinal evaluation of early Alzheimer's disease using brain perfusion...
@F:input.pdf @N: 405   @C:  Near-infrared spectroscopy can detect brain activity...
\end{Verbatim}

The parameters in outputs with following meaning:

\begin{Verbatim}[commandchars=\\\{\}]
@F: prefix output lines with filename.
@N: prefix output lines with line number within pdf text.
@C: indicate the context of matches.
\end{Verbatim}


\section{3. pdftoc.py}
\label{utility:pdftoc-py}
Coming soon...


\section{4. dictviewer.py}
\label{utility:dictviewer-py}
Coming soon...


\chapter{PDFDig Release}
\label{release:pdfdig-release}\label{release::doc}

\section{PDFDig 0.2 (released 2012-04-19)}
\label{release:pdfdig-0-2-released-2012-04-19}
This is a update release.

\textbf{New Features}
\begin{itemize}
\item {} 
Match: Support sentence-based context of matches.

\item {} 
Match: Support highlight the matches.

\item {} 
pdfgrep: Support search all files under each directory.

\item {} 
pdfgrep: Add highlight option.

\end{itemize}

\textbf{Fixes}
\begin{itemize}
\item {} 
Text: Fix cross-platform check for pdftotext

\end{itemize}


\section{PDFDig 0.1 (released 2012-04-10)}
\label{release:pdfdig-0-1-released-2012-04-10}
This is the initial release.

\textbf{New Features}
\begin{itemize}
\item {} 
Text: Convert PDF to text using `pdftotext' and normalize the text. Store text lines as an list object.

\item {} 
Match: Pattern matching based on Text. Store matches as an list object.

\item {} 
TOC: Build the Table of Content(TOC) of PDF document, filtering by a provided TOC dictionary.

\item {} 
pdftotext: a (Command Line Interface) CLI utility based on Text.

\item {} 
pdfgrep: a CLI utility based on Match.

\item {} 
pdftoc: a CLI utility based on TOC

\end{itemize}

Refer to {\hyperref[intro::doc]{\emph{Introduction}}} to see the details of New Features.


\chapter{Indices and tables}
\label{index:indices-and-tables}\begin{itemize}
\item {} 
\emph{genindex}

\item {} 
\emph{modindex}

\item {} 
\emph{search}

\end{itemize}



\renewcommand{\indexname}{Index}
\printindex
\end{document}
